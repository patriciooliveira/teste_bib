%versão 0.01
\documentclass[a4paper]{article}
\usepackage[english]{babel}
\usepackage[utf8x]{inputenc}
\usepackage{amsmath}
\usepackage{graphicx}
\usepackage[colorinlistoftodos]{todonotes}

\title{Utilização do ESP8266}
\author{Josiel Patricio Pereira de Oliveira}

\begin{document}
\maketitle

%\begin{abstract}
%Your abstract.
%\end{abstract}

\section{Introdução}

\paragraph{}
Este trabalho tem como objetivo servir de material introdutório à utilização e programação do chip ESP8266 como microcontrolador e não  apenas 
como periférico de acesso à redes sem fio. Tal material é destinados aos alunos ingressantes  no projeto de Automação Residencial via padrões abertos.

\section{IDE} 

\paragraph{}
Ambiente de desenvolvimento integrado
software que agrega em si editores de texto para edição do código fonte um compilador que traduz o código fonte para linguagem de máquina e no caso de IDE utilizada será a da plataforma arduino.


\subsection{Configuração da IDE}
Para adicionar as placas referentes aos módulos produzidos com o ESP8266 é préciso seguir os seguintes passos:
\begin{enumerate}

\item Clicar em Arquivo → Preferências (ou simplesmente Ctrl + “,”)
\item No campo URLs Adicionais de Gerenciadores de Placa: digitar\\
\verb|http://arduino.esp8266.com/stable/package_esp8266com_index.json|
clicar em OK 
\item Clicar em Ferramentas → Placa → Gerenciador de Placas \dots 
\item Na nova caixa Gerenciador de Placas, rolar a barra de rolagem até encontrar esp8266.
\item Selecionar a versão e clicar em instalar.
\end{enumerate}
%explipar utilização da ide e as configurações imagens 

\section{componentes de Hardware}
Para utilização de módulos não são necessários mais que quatro resistores e um capacitor, a depender do módulo, além da fonte de energia claro.
Porém para a gravação é preciso de algum componentes de hardware adicionais como por exemplo um conversos serial ou um gravador para que seja possível a comunicação entre o microcontrolador e a máquina que está rodando o software de gravação.

\subsection{Gravando e executando o primeiro programa}
\begin{enumerate}
\item Conversor USB-UART
\item Alimentação 3,3V
\item Módulo ESP.
\item resistor 10K$\Omega$
\end{enumerate}

\begin{figure}
\centering
\includegraphics[width=1\textwidth]{conexao_minima.png}
\end{figure}

http://www.fritzenlab.com.br/2016/03/wifi-esp12-esp8266-programando-com-ide.html

http://br-arduino.org/2015/08/nodemcu-esp8266.html

\end{document}
